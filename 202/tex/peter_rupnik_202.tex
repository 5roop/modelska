
\documentclass[a4paper,oneside,12pt]{article}

\usepackage[utf8]{inputenc}    % make čšž work on input
\usepackage[T1]{fontenc}       % make čšž work on output
\usepackage[slovene]{babel}    % slovenian language and hyphenation
\usepackage[reqno]{amsmath}    % basic math
\usepackage{amssymb,amsthm}    % math symbols and theorem environments
\usepackage{graphicx}          % images
\usepackage{enumerate}
\usepackage{subcaption}
\usepackage{graphicx}
\usepackage[
  paper=a4paper,
  top=2.5cm,
  bottom=2.5cm,
  left=2.5cm,
  right=2.5cm
% textheight=24cm,
]{geometry}  % page geomerty

% vstavi svoje pakete tukaj
\usepackage{fancyhdr}
\usepackage{harpoon}
\usepackage{tikz}
\usepackage{makeidx}
\makeindex
\usepackage[all]{xy}
\usepackage{minted}
  % algorithms
  % \RequirePackage{algpseudocode}  % za psevdokodo
  % \RequirePackage{algorithm}      % za plovke
  % \floatname{algorithm}{Algoritem}
  % \renewcommand{\listalgorithmname}{Kazalo algoritmov}



% clickable references, pdf toc
\usepackage[bookmarks, colorlinks=true, linkcolor=black, anchorcolor=black,
  citecolor=black, filecolor=black, menucolor=black, runcolor=black,
  urlcolor=black, pdfencoding=unicode]{hyperref}



% stavi lastne definicije tukaj
\newcommand{\dpar}[2]{\frac{\partial #1}{\partial #2}}
% \newcommand{\HH}{$\mathcal{H}$}
% \newcommand{\LL}{$\vec{\ell}$}
% \newcommand{\AA}{$\vec{{A}}$}
  % \floatname{listing}{Koda}
  % \renewcommand{\listalgorithmname}{Kazalo programske kode}
% \pagestyle{empty}              % vse strani prazne (ni okraskov, številčenja....)
 \setlength{\parindent}{0pt}    % zamik vsakega odstavka
\setlength{\parskip}{10pt}     % prazen prostor pod odstavkom
%\setlength{\overfullrule}{30pt}  % oznaci predlogo vrstico z veliko črnine
\frenchspacing % to se priporoča, da bo presledek za piko na koncu stavka enako dolg kot obični.
%Alternativa je ročno metanje \ za vsako okrajšavo.
\usepackage{braket}
\usepackage{todonotes}
\hypersetup{pdftitle={202: Navadne diferencialne enačbe: robni problemi}, pdfauthor={Peter Rupnik}} %To piše v pdf viewerju na vrhu! veri neat
\title{202: Navadne diferencialne enačbe: robni problemi}
\author{Peter Rupnik\\28182021}
\begin{document}
\maketitle

\begin{enumerate}

    \item Razišči najugodnejši postopek za reševanje radialnega dela
      Schrödingerjeve enačbe s Coulombskim potencialom. Za vodikov atom
      ima enačba obliko
    %
    \begin{equation*}
    \left[-\frac{\mathrm{d}^2}{\mathrm{d}x^2}-\frac{2}{x}+\frac{\ell(\ell+1)}{x^2}-e\right]R(x)=0,
    \qquad R(0)=0,\quad R(\infty)=0
    \end{equation*}
    %
    če je $\Psi(x)=R(x)/x$, $x=r/a$, $e=E/E_0$, $a$ je Bohrov radij in
    $E_0=\SI{13.6}{eV}$.

    Uporabiš lahko metodo Runge-Kutta, še boljša pa je metoda {\sl
      Numerova}, ki jo priporočajo za ta tip problemov in je za red boljša
    od Runge-Kutta 4.~reda. Pri tej metodi enačbo
    %
    \begin{equation*}
    \left[\frac{\mathrm{d}^2}{\mathrm{d}x^2}+k^2(x)\right]y(x)=0
    \end{equation*}
    %
    zapišemo kot
    %
    \begin{equation*}
    \left(1+\frac{h^2}{12}k^2_{i+1}\right)y_{i+1}-
    2\left(1-\frac{5h^2}{12}k^2_i\right)y_i+
    \left(1+\frac{h^2}{12}k^2_{i-1}\right)y_{i-1}=0+{\cal O}(h^6)
    \end{equation*}

    Natančnost metod lahko preveriš s točnimi (nenormaliziranimi)
    lastnimi funkcijami
    %
    \begin{align*}
    R_{n=1,\ell=0}(x)&=xe^{-x}\\
    R_{20}(x)&=x(1-\tfrac{1}{2}x)e^{-x/2}\\
    R_{21}(x)&=x^2e^{-x/2}
    \end{align*}

    \item Za propagacijo monokromatske svetlobe v snovi velja Helmholtzova
      enačba, ki se v brezdimenzijski obliki glasi
      \begin{equation*}
        (\nabla^2+n({r})^2k^2)\Psi({r})=0.
      \end{equation*}
    Pri tem je $n({r})$ lomni količnik, $k$ pa brezdimenzijsko valovno
    število. Pri iskanju lastnih načinov valovanja v svetlobnih vlaknih
    uporabimo nastavek $\Psi(x)=(R(x)/\sqrt{x}) e^{i \lambda z}$. Za
    radialno simetrična stanja dobimo enačbo
      \begin{equation*}
        \left[\Dd[2]{}{x}+\frac{1}{4x^2}+n(x)^2 k^2-\lambda^2\right]R(x)=0.
      \end{equation*}
    To enačbo lahko rešujemo z metodo {\sl Numerova}, pri čemer vzdolžno valovno
    število $\lambda$ igra vlogo lastne vrednosti.

    Za profil lomnega količnika
    \begin{equation*}
      n(x)=\begin{cases}2-\frac12x^2 & x<1\\ 1 & x\geq 1\end{cases}
    \end{equation*}
    izračunaj disperzijsko relacijo $\lambda(k)$ za $0.8<k<10$ in ugotovi,
    pri katerih valovnih številih $k$ obstaja samo eno vezano stanje
    (t.i.~\emph{single mode fiber}).

    \end{enumerate}
\section{1. naloga}
\subsection{Tipanje problema}
Obravnave se lotim z določanjem reprezentacije cevi v matrični obliki. Dobim spodnjo sliko. Poleg geometrijskih atributov sem na koncu zahteval še `črn rob' okrog cevi, da verno opišem problem. Število točk v vsaki dimenziji znaša 27.
\begin{center}
    \includegraphics[width=0.5\textwidth]{../old/1-hiska.pdf}
\end{center}
Namesto reševanja sistema diskretnih diferencialnih enačb v matrični obliki sem se lotil poskušanja z iterativnimi metodami. Najprej sem poskusil Jacobijevo metodo na kvadratni cevi, da se uverim, ali je problem zastavljen pravilno:
\begin{center}
    \begin{minipage}{0.45\textwidth}
        \centering
    \includegraphics[width=\textwidth]{../old/../old/0-kvadratna_i3.pdf}
    {Hitrostno polje v kvadratni cevi po treh korakih iteracije.}
    \end{minipage}\hfill
    \begin{minipage}{0.45\textwidth}
        \centering
        \includegraphics[width=1\textwidth]{../old/0-kvadratna_i50.pdf}
    {Hitrostno polje v kvadratni cevi po 50 korakih iteracije. Maksimalna hitrost se je povečala, kar lahko kaže na slabo zastavljen postopek.}
    \end{minipage}


    \begin{minipage}{0.45\textwidth}
        \centering
    \includegraphics[width=\textwidth]{../old/0-kvadratna_i100.pdf}
    {Hitrostno polje v kvadratni cevi po 100 korakih iteracije.}
    \end{minipage}\hfill
    \begin{minipage}{0.45\textwidth}
        \centering
        \includegraphics[width=1\textwidth]{../old/0-kvadratna_i200.pdf}
    {Hitrostno polje v kvadratni cevi po 200 korakih iteracije. }
    \end{minipage}

    \begin{minipage}{0.45\textwidth}
        \centering
    \includegraphics[width=\textwidth]{../old/0-kvadratna_i600.pdf}
    {Hitrostno polje v kvadratni cevi po 600 korakih iteracije. Z metodo ostrega pogleda sem ocenil, da smo okrog iteracije 400 dosegli konvergenco hitrostnega polja.}
    \end{minipage}\hfill
    \begin{minipage}{0.45\textwidth}
        \centering
        \includegraphics[width=1\textwidth]{../old/0-kvadratna_i1200.pdf}
    {Hitrostno polje v kvadratni cevi po 1200 korakih iteracije. Tudi po trikrat prekoračeni ocenjeni točki konvergence je hitrostno polje konstantno.}
    \end{minipage}
\end{center}
Validacija rezultatov je na tej točki predvsem optična, zato velja za primerjavo metod v nadaljevanju uvesti primerne metrike, denimo relativno razliko med iteracijami, ali pa kar Poiseuillov koeficient. Ker je slednja opcija zahtevnejša, sem implementiral preprosto funkcijo:
\begin{python}
def rel_err(a1: numpy.ndarray, a2: numpy.ndarray) -> float:

    return np.sum(np.abs(a1-a2).reshape(-1))/np.sum(np.abs(a1.reshape(-1)))
\end{python}
Opazim, da sprva hitro približevanje konvergenci preide v počasnejše, a monotono padanje vse do približno iteracije številka 600, ko postaneta zaporedna hitrostna polja do strojne natančnosti enaka.
\begin{center}
    \includegraphics[width=0.5\textwidth]{../old/0-errors.pdf}
\end{center}
Nadaljujem lahko s podanim profilom cevi.
\subsection{Rešitev za podan profil cevi}
S pripravljeno matriko za profil cevi lahko postopek za kvadratno cev hitro popravim za dano cev. Nadaljnja metodologija je enaka.
\begin{center}
    \begin{minipage}{0.45\textwidth}
        \centering
    \includegraphics[width=\textwidth]{../old/1-hiska_i3.pdf}
    {Hitrostno polje v podani cevi po treh korakih iteracije.}
    \end{minipage}\hfill
    \begin{minipage}{0.45\textwidth}
        \centering
        \includegraphics[width=1\textwidth]{../old/1-hiska_i50.pdf}
    {Hitrostno polje v podani cevi po 50 korakih iteracije. Opazimo enake tendence povečevanja hitrostnega polja kot pri okrogli cevi.}
    \end{minipage}

    \begin{minipage}{0.45\textwidth}
        \centering
    \includegraphics[width=\textwidth]{../old/1-hiska_i100.pdf}
    {Hitrostno polje v podani cevi po 100 korakih iteracije.}
    \end{minipage}\hfill
    \begin{minipage}{0.45\textwidth}
        \centering
        \includegraphics[width=1\textwidth]{../old/1-hiska_i200.pdf}
    {Hitrostno polje v podani cevi po 200 korakih iteracije. }
    \end{minipage}

    \begin{minipage}{0.45\textwidth}
        \centering
    \includegraphics[width=\textwidth]{../old/1-hiska_i600.pdf}
    {Hitrostno polje v podani cevi po 600 korakih iteracije.}
    \end{minipage}\hfill
    \begin{minipage}{0.45\textwidth}
        \centering
        \includegraphics[width=1\textwidth]{../old/1-hiska_i1200.pdf}
    {Hitrostno polje v kvadratni cevi po 1200 korakih iteracije.}
    \end{minipage}
\end{center}
Spet lahko pogledamo hitrost konvergence. Kot je razvidno iz točke, ko razlika med zaporednima iteracijama pade na 0, je v trenutnem primeru konvergenca hitrejša.
\begin{center}
    \includegraphics[width=0.5\textwidth]{../old/1-errors.pdf}
\end{center}
\subsection{Implementacija pospešene iteracijske sheme}
S predavanj vemo, da bi Gauss-Seidlov postopek izboljšal hitrost konvergence le za konstanten faktor, zato sem raje nadaljeval z implementacijo metode~SOR. Popravljena koda je delovala odlično, evalvacija pa mi je povzročala nekaj težav. Končno sem se odločil za sledeč postopek: generiral sem `pravo' hitrostno polje s 1000 koraki iteracije, nato pa sem pri posameznih vrednostih $\omega$ z metodo SOR iteriral le 100 korakov. Rezultata sem primerjal z vsoto absolutnih razlik po elementih.
\begin{center}
    \includegraphics[width=0.7\textwidth]{../old/1-omegas.pdf}
\end{center}
Kot smo pričakovali s predavanj, je optimalna $\omega$ odvisna od dimenzije problema. Naraščanje napake (neodvisno od $\omega$) lahko pojasnimo s tem, da je matrika hitrostnega polja večja, zaradi česar je tudi razlika med `pravim' in trenutnim hitrostnim poljem večja.

Nadaljujem z uvedbo parametra $\alpha$, kot je definiran v navodilih:
\[\omega = \frac{2}{1+\alpha \frac{\pi}{N}}.\]
Kot razvidno na sliki spodaj, je zdaj optimalen pospešek invarianten od števila točk na mreži.
\begin{center}
    \includegraphics[width=0.7\textwidth]{../old/1-alphas.pdf}
\end{center}
\subsection{Poiseuillov koeficient}
Zdaj lahko z optimalnim pospeškom iščem Poiseuillov koeficient za dano cev. Zanj velja:

\[ \Phi_V = \iint u(x,y) \; \mathrm{d}S = \frac{C S^2}{8 \pi \eta} \dpar{p}{z}.\]

Izrazim lahko \[ C^{*} = C \eta^{-1} \dpar{p}{z},\] za presek cevi pa vzamem analitično vrednost $S = \dfrac{5}{9}.$

Za izračun $\Phi$ sem poskusil integracijo z dvodimenzionalnim Simposonom, s svojo implementacijo dvodimenzionalne verzije metode za trapezijsko integracijo \texttt{numpy.trapz}, pa tudi z najbolj rudimentarno sumacijo. Medsebojno so vse metode dosegle natančnosti pod $0.1 \%$. Pri iteraciji sem uporabil zgoraj določen optimalen parameter $\alpha$ za pospešek iteracije.

\begin{center}
    \includegraphics[width=0.9\textwidth]{../old/1-C.pdf}


    {Razvoj Poiseuillovega koeficienta skozi iteracijo. Opazimo hitro stabilizacijo vrednosti, po kateri so popravki venomer manjši. Diskretizacija mreže je v vsaki dimenziji znašala 120 točk.}
\end{center}


\clearpage
\section{2. naloga}

\subsection{Reševanje}

Ponovno pripravim matriko, ki v diskretni obliki opisuje operator $\nabla^2$. Tokrat ni simetrična, zato sem pri uporabi \texttt{scipy.sparse.linalg} našel kompleksne rešitve (sicer z imaginarno komponento, primerljivo z $\varepsilon_\text{machine}$). S knjižnjico \texttt{scipy.linalg} teh problemov nisem imel, zato sem se posluževal slednje. Na podlagi rezultatov prve naloge pričakujem, da bi se metodi dobro ujemali in bo nadaljnje obnašanje rešitev podobno.

Ko sem lastne vektorje in vrednosti izračunal, sem si jih izrisal v polarni geometriji.

\begin{center}
    \includegraphics[width=0.9\textwidth]{../old/2-nihajni_nacini.pdf}
\end{center}
Ker me je skrbelo, zakaj  ne vidim večje polarne odvisnosti, sem pogledal tudi višje lastne nihajne načine in se uveril, da dobim pričakovano obnašanje tudi po polarnem kotu.



\begin{center}
    \includegraphics[width=0.8\textwidth]{../old/2-nihajni_nacini_miks.pdf}
\end{center}
Dobljen spekter lastnih vrednosti prikazujem spodaj.
\begin{center}
\includegraphics[width=0.7\textwidth]{../old/2-spekter.pdf}
\end{center}


\begin{thebibliography}{9}
\bibitem{wiki} \url{https://en.wikipedia.org/wiki/Inverse_iteration}

\end{thebibliography}
\end{document}

\begin{figure}
    \centering
        \begin{minipage}{0.45\textwidth}
        \centering
    \includegraphics[width=\textwidth]{../old/}
    \caption{}
        \label{fig:}
    \end{minipage}\hfill
    \begin{minipage}{0.45\textwidth}
        \centering
        \includegraphics[width=1\textwidth]{../old/}
    \caption{}
    \label{fig:}
    \end{minipage}
\end{figure}

\end{document}