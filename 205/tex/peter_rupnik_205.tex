
\documentclass[a4paper,oneside,12pt]{article}

\usepackage[utf8]{inputenc}    % make čšž work on input
\usepackage[T1]{fontenc}       % make čšž work on output
\usepackage[slovene]{babel}    % slovenian language and hyphenation
\usepackage[reqno]{amsmath}    % basic math
\usepackage{amssymb,amsthm}    % math symbols and theorem environments
\usepackage{graphicx}          % images
\usepackage{enumerate}
\usepackage{physics}
\usepackage[
  paper=a4paper,
  top=2.5cm,
  bottom=2.5cm,
  left=2.5cm,
  right=2.5cm
% textheight=24cm,
]{geometry}  % page geomerty

% vstavi svoje pakete tukaj
\usepackage{fancyhdr}
\usepackage{tikz}
\usepackage{makeidx}
\makeindex
\usepackage[all]{xy}
\usepackage{pythonhighlight}
% \usepackage{minted}
  % algorithms
%   \RequirePackage{algpseudocode}  % za psevdokodo
%   \RequirePackage{algorithm}      % za plovke
%   \floatname{algorithm}{Algoritem}
%   \renewcommand{\listalgorithmname}{Kazalo algoritmov}
%  \algnewcommand\algorithmicto{\textbf{to}}
%  \algnewcommand\algorithmicin{\textbf{in}}
%  \algnewcommand\algorithmicforeach{\textbf{for each}}
%  \algrenewtext{For}[3]{\algorithmicfor\ #1 $\gets$ #2\ \algorithmicto\ #3\ \algorithmicdo}
%\algdef{S}[FOR]{ForEach}[2]{\algorithmicforeach\ #1\ \algorithmicin\ #2\ \algorithmicdo}


% clickable references, pdf toc
\usepackage[bookmarks, colorlinks=true, linkcolor=black, anchorcolor=black,
  citecolor=black, filecolor=black, menucolor=black, runcolor=black,
  urlcolor=black, pdfencoding=unicode]{hyperref}

\setlength{\parindent}{0pt}    % zamik vsakega odstavka
\setlength{\parskip}{10pt}     % prazen prostor pod odstavkom
% lastne definicije
\newcommand{\N}{\ensuremath{\mathbb{N}}}
\newcommand{\Z}{\ensuremath{\mathbb{Z}}}
\newcommand{\Q}{\ensuremath{\mathbb{Q}}}
\newcommand{\R}{\ensuremath{\mathbb{R}}}
\renewcommand{\C}{\ensuremath{\mathbb{C}}}
% stavi lastne definicije tukaj
%   \floatname{listing}{Koda}
%   \renewcommand{\listalgorithmname}{Kazalo programske kode}
% \pagestyle{empty}              % vse strani prazne (ni okraskov, številčenja....)
% \setlength{\parindent}{0pt}    % zamik vsakega odstavka
% \setlength{\parskip}{10pt}     % prazen prostor pod odstavkom
%\setlength{\overfullrule}{30pt}  % oznaci predlogo vrstico z veliko črnine
\frenchspacing % to se priporoča, da bo presledek za piko na koncu stavka enako dolg kot obični.
%Alternativa je ročno metanje \ za vsako okrajšavo.
\newcommand{\dpar}[2]{\frac{\partial #1}{\partial #2}}
\newcommand{\td}[2]{\frac{\mathrm{d}{ #1}}{\mathrm{d}{ #2}}}
\hypersetup{pdftitle={Modelska 205}, pdfauthor={Peter Rupnik}} %To piše v pdf viewerju na vrhu! veri neat
\title{205: Parcialne diferencialne enačbe: robni problemi in relaksacija}
\author{Peter Rupnik}
\def\thesection{\arabic{section}. naloga}
\def\thesubsection{\arabic{section}.\arabic{subsection}}
\begin{document}
\maketitle

\begin{enumerate}

    \item Razišči najugodnejši postopek za reševanje radialnega dela
      Schrödingerjeve enačbe s Coulombskim potencialom. Za vodikov atom
      ima enačba obliko
    %
    \begin{equation*}
    \left[-\frac{\mathrm{d}^2}{\mathrm{d}x^2}-\frac{2}{x}+\frac{\ell(\ell+1)}{x^2}-e\right]R(x)=0,
    \qquad R(0)=0,\quad R(\infty)=0
    \end{equation*}
    %
    če je $\Psi(x)=R(x)/x$, $x=r/a$, $e=E/E_0$, $a$ je Bohrov radij in
    $E_0=\SI{13.6}{eV}$.

    Uporabiš lahko metodo Runge-Kutta, še boljša pa je metoda {\sl
      Numerova}, ki jo priporočajo za ta tip problemov in je za red boljša
    od Runge-Kutta 4.~reda. Pri tej metodi enačbo
    %
    \begin{equation*}
    \left[\frac{\mathrm{d}^2}{\mathrm{d}x^2}+k^2(x)\right]y(x)=0
    \end{equation*}
    %
    zapišemo kot
    %
    \begin{equation*}
    \left(1+\frac{h^2}{12}k^2_{i+1}\right)y_{i+1}-
    2\left(1-\frac{5h^2}{12}k^2_i\right)y_i+
    \left(1+\frac{h^2}{12}k^2_{i-1}\right)y_{i-1}=0+{\cal O}(h^6)
    \end{equation*}

    Natančnost metod lahko preveriš s točnimi (nenormaliziranimi)
    lastnimi funkcijami
    %
    \begin{align*}
    R_{n=1,\ell=0}(x)&=xe^{-x}\\
    R_{20}(x)&=x(1-\tfrac{1}{2}x)e^{-x/2}\\
    R_{21}(x)&=x^2e^{-x/2}
    \end{align*}

    \item Za propagacijo monokromatske svetlobe v snovi velja Helmholtzova
      enačba, ki se v brezdimenzijski obliki glasi
      \begin{equation*}
        (\nabla^2+n({r})^2k^2)\Psi({r})=0.
      \end{equation*}
    Pri tem je $n({r})$ lomni količnik, $k$ pa brezdimenzijsko valovno
    število. Pri iskanju lastnih načinov valovanja v svetlobnih vlaknih
    uporabimo nastavek $\Psi(x)=(R(x)/\sqrt{x}) e^{i \lambda z}$. Za
    radialno simetrična stanja dobimo enačbo
      \begin{equation*}
        \left[\Dd[2]{}{x}+\frac{1}{4x^2}+n(x)^2 k^2-\lambda^2\right]R(x)=0.
      \end{equation*}
    To enačbo lahko rešujemo z metodo {\sl Numerova}, pri čemer vzdolžno valovno
    število $\lambda$ igra vlogo lastne vrednosti.

    Za profil lomnega količnika
    \begin{equation*}
      n(x)=\begin{cases}2-\frac12x^2 & x<1\\ 1 & x\geq 1\end{cases}
    \end{equation*}
    izračunaj disperzijsko relacijo $\lambda(k)$ za $0.8<k<10$ in ugotovi,
    pri katerih valovnih številih $k$ obstaja samo eno vezano stanje
    (t.i.~\emph{single mode fiber}).

    \end{enumerate}
\section{}
Nalogo bom poskušal reševati z metodo Numerova. Za inicializatijo iterativne sheme potrebujem $y_0$ in $y_1$. Četudi lahko za $y_0$ uporabnim robni pogoj $R\left(x=0\right) = 0$, bi moral naslednjo točko streljati. Zagati se lahko izognemo, če $y$ zapišemo kot potenčno vrsto:

\begin{equation*}
    y = a_0 + a_1 x + a_2 x^2 + \cdots
\end{equation*}
Drugi odvod lahko aproksimiramo z
\begin{equation*}
    y_i'' = \frac{y_{i+1} + y_{i-1} - 2 y_i}{h^2} + \mathcal{O} (h^2)
\end{equation*}

\section{}
\end{document}