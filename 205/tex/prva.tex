\subsection{Tipanje problema}
Obravnave se lotim z določanjem reprezentacije cevi v matrični obliki. Dobim spodnjo sliko. Poleg geometrijskih atributov sem na koncu zahteval še `črn rob' okrog cevi, da verno opišem problem. Število točk v vsaki dimenziji znaša 27.
\begin{center}
    \includegraphics[width=0.5\textwidth]{../old/1-hiska.pdf}
\end{center}
Namesto reševanja sistema diskretnih diferencialnih enačb v matrični obliki sem se lotil poskušanja z iterativnimi metodami. Najprej sem poskusil Jacobijevo metodo na kvadratni cevi, da se uverim, ali je problem zastavljen pravilno:
\begin{center}
    \begin{minipage}{0.45\textwidth}
        \centering
    \includegraphics[width=\textwidth]{../old/../old/0-kvadratna_i3.pdf}
    {Hitrostno polje v kvadratni cevi po treh korakih iteracije.}
    \end{minipage}\hfill
    \begin{minipage}{0.45\textwidth}
        \centering
        \includegraphics[width=1\textwidth]{../old/0-kvadratna_i50.pdf}
    {Hitrostno polje v kvadratni cevi po 50 korakih iteracije. Maksimalna hitrost se je povečala, kar lahko kaže na slabo zastavljen postopek.}
    \end{minipage}


    \begin{minipage}{0.45\textwidth}
        \centering
    \includegraphics[width=\textwidth]{../old/0-kvadratna_i100.pdf}
    {Hitrostno polje v kvadratni cevi po 100 korakih iteracije.}
    \end{minipage}\hfill
    \begin{minipage}{0.45\textwidth}
        \centering
        \includegraphics[width=1\textwidth]{../old/0-kvadratna_i200.pdf}
    {Hitrostno polje v kvadratni cevi po 200 korakih iteracije. }
    \end{minipage}

    \begin{minipage}{0.45\textwidth}
        \centering
    \includegraphics[width=\textwidth]{../old/0-kvadratna_i600.pdf}
    {Hitrostno polje v kvadratni cevi po 600 korakih iteracije. Z metodo ostrega pogleda sem ocenil, da smo okrog iteracije 400 dosegli konvergenco hitrostnega polja.}
    \end{minipage}\hfill
    \begin{minipage}{0.45\textwidth}
        \centering
        \includegraphics[width=1\textwidth]{../old/0-kvadratna_i1200.pdf}
    {Hitrostno polje v kvadratni cevi po 1200 korakih iteracije. Tudi po trikrat prekoračeni ocenjeni točki konvergence je hitrostno polje konstantno.}
    \end{minipage}
\end{center}
Validacija rezultatov je na tej točki predvsem optična, zato velja za primerjavo metod v nadaljevanju uvesti primerne metrike, denimo relativno razliko med iteracijami, ali pa kar Poiseuillov koeficient. Ker je slednja opcija zahtevnejša, sem implementiral preprosto funkcijo:
\begin{python}
def rel_err(a1: numpy.ndarray, a2: numpy.ndarray) -> float:

    return np.sum(np.abs(a1-a2).reshape(-1))/np.sum(np.abs(a1.reshape(-1)))
\end{python}
Opazim, da sprva hitro približevanje konvergenci preide v počasnejše, a monotono padanje vse do približno iteracije številka 600, ko postaneta zaporedna hitrostna polja do strojne natančnosti enaka.
\begin{center}
    \includegraphics[width=0.5\textwidth]{../old/0-errors.pdf}
\end{center}
Nadaljujem lahko s podanim profilom cevi.
\subsection{Rešitev za podan profil cevi}
S pripravljeno matriko za profil cevi lahko postopek za kvadratno cev hitro popravim za dano cev. Nadaljnja metodologija je enaka.
\begin{center}
    \begin{minipage}{0.45\textwidth}
        \centering
    \includegraphics[width=\textwidth]{../old/1-hiska_i3.pdf}
    {Hitrostno polje v podani cevi po treh korakih iteracije.}
    \end{minipage}\hfill
    \begin{minipage}{0.45\textwidth}
        \centering
        \includegraphics[width=1\textwidth]{../old/1-hiska_i50.pdf}
    {Hitrostno polje v podani cevi po 50 korakih iteracije. Opazimo enake tendence povečevanja hitrostnega polja kot pri okrogli cevi.}
    \end{minipage}

    \begin{minipage}{0.45\textwidth}
        \centering
    \includegraphics[width=\textwidth]{../old/1-hiska_i100.pdf}
    {Hitrostno polje v podani cevi po 100 korakih iteracije.}
    \end{minipage}\hfill
    \begin{minipage}{0.45\textwidth}
        \centering
        \includegraphics[width=1\textwidth]{../old/1-hiska_i200.pdf}
    {Hitrostno polje v podani cevi po 200 korakih iteracije. }
    \end{minipage}

    \begin{minipage}{0.45\textwidth}
        \centering
    \includegraphics[width=\textwidth]{../old/1-hiska_i600.pdf}
    {Hitrostno polje v podani cevi po 600 korakih iteracije.}
    \end{minipage}\hfill
    \begin{minipage}{0.45\textwidth}
        \centering
        \includegraphics[width=1\textwidth]{../old/1-hiska_i1200.pdf}
    {Hitrostno polje v kvadratni cevi po 1200 korakih iteracije.}
    \end{minipage}
\end{center}
Spet lahko pogledamo hitrost konvergence. Kot je razvidno iz točke, ko razlika med zaporednima iteracijama pade na 0, je v trenutnem primeru konvergenca hitrejša.
\begin{center}
    \includegraphics[width=0.5\textwidth]{../old/1-errors.pdf}
\end{center}
\subsection{Implementacija pospešene iteracijske sheme}
S predavanj vemo, da bi Gauss-Seidlov postopek izboljšal hitrost konvergence le za konstanten faktor, zato sem raje nadaljeval z implementacijo metode~SOR. Popravljena koda je delovala odlično, evalvacija pa mi je povzročala nekaj težav. Končno sem se odločil za sledeč postopek: generiral sem `pravo' hitrostno polje s 1000 koraki iteracije, nato pa sem pri posameznih vrednostih $\omega$ z metodo SOR iteriral le 100 korakov. Rezultata sem primerjal z vsoto absolutnih razlik po elementih.
\begin{center}
    \includegraphics[width=0.7\textwidth]{../old/1-omegas.pdf}
\end{center}
Kot smo pričakovali s predavanj, je optimalna $\omega$ odvisna od dimenzije problema. Naraščanje napake (neodvisno od $\omega$) lahko pojasnimo s tem, da je matrika hitrostnega polja večja, zaradi česar je tudi razlika med `pravim' in trenutnim hitrostnim poljem večja.

Nadaljujem z uvedbo parametra $\alpha$, kot je definiran v navodilih:
\[\omega = \frac{2}{1+\alpha \frac{\pi}{N}}.\]
Kot razvidno na sliki spodaj, je zdaj optimalen pospešek invarianten od števila točk na mreži.
\begin{center}
    \includegraphics[width=0.7\textwidth]{../old/1-alphas.pdf}
\end{center}
\subsection{Poiseuillov koeficient}
Zdaj lahko z optimalnim pospeškom iščem Poiseuillov koeficient za dano cev. Zanj velja:

\[ \Phi_V = \iint u(x,y) \; \mathrm{d}S = \frac{C S^2}{8 \pi \eta} \dpar{p}{z}.\]

Izrazim lahko \[ C^{*} = C \eta^{-1} \dpar{p}{z},\] za presek cevi pa vzamem analitično vrednost $S = \dfrac{5}{9}.$

Za izračun $\Phi$ sem poskusil integracijo z dvodimenzionalnim Simposonom, s svojo implementacijo dvodimenzionalne verzije metode za trapezijsko integracijo \texttt{numpy.trapz}, pa tudi z najbolj rudimentarno sumacijo. Medsebojno so vse metode dosegle natančnosti pod $0.1 \%$. Pri iteraciji sem uporabil zgoraj določen optimalen parameter $\alpha$ za pospešek iteracije.

\begin{center}
    \includegraphics[width=0.9\textwidth]{../old/1-C.pdf}


    {Razvoj Poiseuillovega koeficienta skozi iteracijo. Opazimo hitro stabilizacijo vrednosti, po kateri so popravki venomer manjši. Diskretizacija mreže je v vsaki dimenziji znašala 120 točk.}
\end{center}

